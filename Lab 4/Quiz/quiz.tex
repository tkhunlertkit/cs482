\documentclass{exam}

\usepackage[margin=1in]{geometry}

\begin{document}

\begin{center}
	\textbf{CS 482 - Rich Internet Applications} \\
	\textbf{Section 802} \\
	\vspace{5mm}
	\makebox[\textwidth]{Name :\enspace\hrulefill} \\
\end{center}


\begin{questions}
	\question[10] Write a valid HTML document which contains an \texttt{unordered list}: \texttt{kale, quinoa, avocado toast}. Beneath the list is a button ``\texttt{add item}". Each time the \texttt{add item} button is pressed, a new item is added to the list. 
	
	\textbf{For up to eight points}: the new item added is always \texttt{cheese curds}. So if \texttt{add item} is clicked \emph{seven} times, the list will contain 10 items, and the last seven items will be cheeses curds.
	
	\textbf{For up to 10 points}: The new items should cycle through \texttt{brats, cheese curds, frozen custard}. So the first time add item is clicked brats is added to the lists, then cheese curds, etc. If the button is clicked five times the list will be:
	\texttt{
		\begin{itemize}
			\item kale
			\item quinoa
			\item avocado toast
			\item brats
			\item cheese curds
			\item frozen custard
			\item brats
			\item cheese curds
		\end{itemize}
	}
	
	All javascript code should be contained in a \texttt{script} element in the \texttt{head} of the document, except that you may use the event attributes (like \texttt{onclick}) of an element to call a function.
\end{questions}

\end{document}